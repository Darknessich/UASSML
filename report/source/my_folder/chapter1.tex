\chapter{Теоретические и методические основы исследования} \label{ch1}

\section{Постановка задачи} \label{ch1:task}
Рассмотрим задачу построения прогностических моделей, направленных на оценку эффективности опрыскивания сельскохозяйственных культур с использованием БПЛА. В рамках данного исследования предполагается, что на эффективность опрыскивания существенное влияние оказывают следующие группы факторов: 
\begin{itemize}
\item характеристики БПЛА, включая параметры его конструкции, высоту полёта, скорость и тип форсунок; 
\item свойства целевой сельскохозяйственной культуры (тип культуры, морфология листьев и пр.); 
\item погодные условия во время обработки (скорость ветра, температура воздуха и влажность).
\end{itemize}

Целевыми переменными в рамках данной задачи выступают:
\begin{itemize}
\item $y_1 \in [0, 100] \subset \mathbb{R}$ --- процент площади покрытия растительности рабочей жидкостью (далее --- \textit{покрытие});
\item $y_2 \in \mathbb{R}_{+}$ --- средний диаметр капель распыления, выраженный в микрометрах (далее --- \textit{размер капель}).
\end{itemize}

Пусть имеется выборка наблюдений, составленная на основе агрегированных и аугментированных данных, извлечённых из открытых научных публикаций \cite{Liu2025, Wu2025}, в которых представлены экспериментальные измерения эффективности опрыскивания при различных конфигурациях БПЛА, типах культур и внешних условиях. Формально, обозначим выборку $\mathcal{D}$ размера $N$ как:

\begin{equation}
	\mathcal{D} = \left\{ \left(\mathbf{x}^{(i)}, y_1^{(i)}, y_2^{(i)}\right) \right\}_{i=1}^{N},
\end{equation}
где каждый $\mathbf{x}^{(i)} \in \mathbb{R}^d$ --- вектор признаков, описывающий конкретную реализацию комбинации параметров:

\begin{equation}
\mathbf{x}^{(i)} = \left[ x_1^{(i)}, x_2^{(i)}, \ldots, x_d^{(i)} \right],
\end{equation}
включающий как числовые переменные (скорость полёта, температура воздуха и т.п.), так и категориальные (тип форсунки, сорт культуры и т.п.). Размерность пространства признаков $d$ определяется количеством отобранных характеристик. Для упрощения формулировки будем считать, что все категориальные признаки закодированы в виде one-hot или target encoding и подлежат обработке с помощью моделей машинного обучения.

Целью работы является построение двух аппроксимирующих моделей $f_1, f_2: \mathbb{R}^d \to \mathbb{R}$, таких что:

\begin{equation}
	\hat{y}_1 = f_1(\mathbf{x}), \quad \hat{y}_2 = f_2(\mathbf{x}),
\end{equation}
где $\hat{y}_1$ и $\hat{y}_2$ --- предсказанные моделью процент покрытия и средний диаметр капель соответственно. Требуется, чтобы модели минимизировали отклонение между предсказанными и фактическими значениями на тестовой выборке с учётом выбранной функции потерь. 

Обучающая выборка сформирована путём объединения результатов независимых  исследований, найденных в научной литературе \cite{Liu2025, Wu2025}, с последующим применением процедур синтетического расширения (data augmentation), включая стохастическое варьирование условий и моделирование шумов измерения.

Таким образом, работа направлена на решение задачи многомерной регрессии с двумя зависимыми переменными, на входе которой --- вектор признаков, характеризующих совокупность агротехнических, технических и метеорологических условий, а на выходе --- количественные оценки параметров эффективности опрыскивания, подлежащие интерпретации в прикладных агроинженерных задачах.

В ходе работы применялись и сравнивались несколько классов моделей: метод множественной линейной регрессии (MLR), регрессия опорных векторов (SVR), градиентный бустинг (GBR), ансамблевые методы --- случайный лес (RF), XGBoost и CatBoost.

\section{Обзор используемых методов} \label{ch1:methods}
\subsection{Множественная линейная регрессия (MLR)}
\subsection{Регрессия опорных векторов (SVR)}
\subsection{Градиентный бустинг (GBR)}
\subsection{Метод случайного леса (RF)}
\subsection{XGBoost}
\subsection{CatBoost}
