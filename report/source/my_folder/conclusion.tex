\chapter*{Заключение} \label{ch-conclusion}
\addcontentsline{toc}{chapter}{Заключение}	% в оглавление 


Современные методы машинного обучения предоставляют инструментарий, позволяющий не только сократить объём полевых исследований, но и повысить точность их оценок. Особенно актуальным становится построение предсказательных моделей на основе агрегированных данных, полученных из открытых источников, включая научные публикации, а также их последующая проверка на предмет пригодности к решению прикладных задач агроинженерии.

В ходе выполнения ВКР была достигнута главная цель - обучение и сравнение регрессионных моделей, предназначенных для прогнозирования эффективности распыления инсектицида с применением БПЛА. При этом эффективность характеризуется площадью покрытия (\%) и диаметром капель (мкм) инсектицида. Для достижения цели были проделаны следующие этапы.

Во-первых, сформирован обучающий датасет, агрегирующий ключевые значения эффективности распыления, полученные из различных независимых источников в открытом доступе. Для повышения полноты и устойчивости выборки была применена процедура аугментации: в частности, использованы методы генерации симметричных распределений и масштабирования.

Во-вторых, на подготовленном датасете были обучены шесть моделей регрессии: множественная линейная регрессия (MLR), регрессия опорных векторов (SVR), метод случайного леса (RF), градиентный бустинг (GBR), XGBoost и CatBoost. Все модели были протестированы на двух ключевых задачах --- предсказание площади покрытия и диаметра капель.

В-третьих, проведён сравнительный анализ точности построенных моделей. Полученные результаты показали, что CatBoost обеспечивает наивысшую обобщающую способность при стабильных метриках как на обучении, так и на тестировании. SVR продемонстрировал высокую эффективность в задаче предсказания диаметра капель, особенно при настройке гиперпараметров. Метод случайного леса и градиентный бустинг показали хорошие, но менее устойчивые результаты, чувствительные к структуре данных. XGBoost проявил признаки переобучения, а линейная регрессия в обоих случаях оказалась неконкурентоспособной, что указывает на нелинейные зависимости в исходных данных.

В-четвёртых, интерпретация результатов позволила выявить чувствительность моделей к неравномерному распределению данных и выбросам, особенно в случае ансамблевых моделей. Также установлено, что модели в целом способны адекватно отражать влияние входных факторов (тип оборудования, давление, рабочая скорость и высота) на характеристики покрытия.

Таким образом, можно сделать общий вывод: агрегированные и аугментированные данные, собранные из открытых источников, обладают высокой прогностической ценностью при условии правильной предварительной обработки. Построенные на их основе модели машинного обучения продемонстрировали высокую точность и применимость к реальным задачам оценки параметров распыления при использовании БПЛА.

На основании полученных результатов можно предложить следующие рекомендации:

\begin{enumerate}
\item При проектировании агрономических экспериментов с использованием БПЛА целесообразно учитывать возможность последующего машинного анализа данных. Для этого важно стандартизировать структуру и формат параметров.
\item Модели CatBoost и SVR рекомендуется использовать в качестве базовых для задач предсказания покрытия и диаметра капель соответственно, с последующей тонкой настройкой под конкретные условия.
\item Для снижения чувствительности к выбросам рекомендуется применять методы предварительной фильтрации данных либо адаптивного взвешивания ошибок при обучении.
\item Системы поддержки агротехнологических решений могут быть дополнены предсказательными модулями, построенными на основе моделей машинного обучения, обученных на объединённых выборках с привлечением открытых научных источников.
\end{enumerate}

В заключение следует подчеркнуть, что развитие цифровых подходов в агрономии, включая машинное обучение, открывает новые возможности для оптимизации процессов защиты растений, повышения эффективности агротехнологий и внедрения точного земледелия. Полученные в данной работе результаты подтверждают практическую применимость современных методов анализа данных в агроинженерной практике и могут служить основой для дальнейших исследований и разработок в этой области.

