\chapter*{Введение} % * не проставляет номер
\addcontentsline{toc}{chapter}{Введение} % вносим в содержание


Современное сельское хозяйство сталкивается с необходимостью повышения эффективности применения инсектицидов при одновременном снижении их негативного воздействия на растения и окружающую среду. В связи с этим особое внимание уделяется технологии обработки сельскохозяйственных культур с применением беспилотных летательных аппаратов (БПЛА) \cite{Kurchenko2023}. Возможность точной и гибкой настройки их параметров оказывает прямое влияние на эффективность доставки препарата к целевым поверхностям. При этом неправильно подобранные характеристики аппарата способны привести к избыточному расходу препарата, неравномерному покрытию, загрязнению окружающей среды и нанесению вреда обрабатываемым культурам \cite{Yang2024}. Построение математических моделей, позволяющих прогнозировать данные параметры на основании входных условий, является важной прикладной задачей, решение которой может существенно повысить результативность агротехнических мероприятий.


На текущий момент существует множество исследований, посвящённых влиянию агротехнических и метеорологических условий на характеристики распыления при применении БПЛА. Например, в работе \cite{Liu2025} рассмотрена задача подбора параметров работы БПЛА для уменьшения дрейфа и увеличения удержания раствора на листьях. Исследование \cite{Wu2025} посвящено анализу условий применения инсектицидов в чайных плантациях. Аналогично была исследована эффективность распыления пестицида на полях с соей и пшеницей при использовании БПЛА с 4-мя типами форсунок \cite{Lopes2023}, на кофейных плантациях с 3-мя разными сортами культуры \cite{Vitoria2022} и в виноградниках \cite{Biglia2022}. Все эти исследования подчёркивают высокую зависимость результатов обработки от множества факторов: скорости и высоты полёта БПЛА, типа культуры, погодных условий и свойств раствора. Так, с биологической точки зрения актуальность работы заключается в необходимости повышения эффективности опрыскивания растений для борьбы с вредителями, болезнями и сорняками, которые сильно влияют на урожайность и качество продукции.


С другой стороны, все исследуемые процессы достаточно сложны для описания и поддаются формализации только с применением современных методов машинного обучения. На протяжении последних лет оно широко используется для анализа сложных, многомерных и зачастую зашумлённых данных. Его быстрое развитие позволяет применять современные алгоритмы анализа для построения предсказательных моделей во многих сферах нашей жизни. Например, в работах \cite{Liakos2018} и \cite{Jordan2015} показано, что машинное обучение находит всё более широкое применение в сельском хозяйстве: от прогнозирования урожайности и определения состояния посевов до задач точного земледелия и оптимизации применения средств защиты растений. Тем не менее, для задачи прогнозирования параметров распыления инсектицида при помощи БПЛА сравнительный анализ эффективности различных моделей машинного обучения остаётся ограниченно представленным в научной литературе.


Объектом исследования является процесс опрыскивания сельскохозяйственных культур инсектицидами с помощью БПЛА. Предметом исследования выступают площадь покрытия (\%) и диаметр капель (мкм), характеризующие эффективность распыления в зависимости от совокупности технических, агротехнических и метеорологических факторов.


Таким образом, цель работы: обучить и сравнить регрессионные модели, предназначенных для прогнозирования эффективности распыления инсектицида с применением БПЛА.


Для её достижения необходимо решить следующие задачи:

\begin{enumerate}
	\item сформировать обучающий датасет на основе агрегированных и аугментированных данных из открытых научных публикаций;
	\item обучить на подготовленном датасете следующие регрессионные модели: множественная линейная регрессия (MLR), регрессия опорных векторов (SVR), метод случайного леса (RF), градиентный бустинг (GBR), XGBoost и CatBoost;
	\item провести оценку точности построенных моделей и сравнить их;
	\item интерпретировать полученные результаты с позиции значимости входных факторов и применимости моделей в реальных условиях агроинженерии.
\end{enumerate}


Для выполнения анализа в практической части ВКР были использованы материалы \cite{Wu2025, Liu2025, Vitoria2022, Lopes2023, Biglia2022}. Из них были извлечены исходные данные для формирования датасета.


Методологической основой ВКР выступают методы машинного обучения, применяемые для построения и анализа регрессионных моделей. В частности: множественная линейная регрессия (MLR) \cite{rao1973}, регрессия опорных векторов (SVR) \cite{vapnik1995, vapnik1997svr}, случайный лес (RF) \cite{Breiman2001}, градиентный бустинг (GBR) \cite{friedman2001, hastie2009elements}, XGBoost \cite{Chen2016} и CatBoost \cite{prokhorenkova2018catboost}. В работе используются методы предобработки и аугментации данных, а также подходы к оценке качества моделей на основе метрик регрессии.



%% Вспомогательные команды - Additional commands
%\newpage % принудительное начало с новой страницы, использовать только в конце раздела
%\clearpage % осуществляется пакетом <<placeins>> в пределах секций
%\newpage\leavevmode\thispagestyle{empty}\newpage % 100 % начало новой строки