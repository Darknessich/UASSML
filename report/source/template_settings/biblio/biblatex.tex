%%% Реализация библиографии пакетами biblatex и biblatex-gost с использованием движка biber %%%

%\usepackage{csquotes} % biblatex рекомендует его подключать. Пакет для оформления сложных блоков цитирования.
%%% Загрузка пакета с основными настройками %%%
\ifnumequal{\value{draft}}{0}{% Чистовик
\usepackage[%
backend=biber,% движок
bibencoding=utf8,% кодировка bib файла
sorting=none,% настройка сортировки списка литературы
style=gost-numeric,% стиль цитирования и библиографии (по ГОСТ)
%%style=gost-authoryear,% стиль цитирования и библиографии (по ГОСТ)
%%%% выставить следующую опцию <<babel>> и закомментировать <<language=english>> для достижения многоязычных ссылок
babel=other, %выставим для отображения разных языков
%%language=english,%только английский = \setlanguage{}; autobib получение языка из babel/polyglossia, default: autobib % если ставить autocite или auto, то цитаты в тексте с указанием страницы, получат указание страницы на языке оригинала
%%autolang=other,%other многоязычная библиография
%%clearlang=true,% внутренний сброс поля language, если он совпадает с языком из babel/polyglossia
defernumbers=true,% нумерация проставляется после двух компиляций, зато позволяет выцеплять библиографию по ключевым словам и нумеровать не из большего списка
sortcites=true,% сортировать номера затекстовых ссылок при цитировании (если в квадратных скобках несколько ссылок, то отображаться будут отсортированно, а не абы как)
doi=true,% Показывать или нет ссылки на DOI
isbn=false% Показывать или нет ISBN, ISSN, ISRN
]{biblatex}[2016/09/17]%
}{%Черновик
\usepackage[%
backend=biber,% движок
bibencoding=utf8,% кодировка bib файла
sorting=none,% настройка сортировки списка литературы
]{biblatex}[2016/09/17]%
}
%%TO-DO: продумать автозамену всех полей hyphenation на language



%%% Подключение файлов bib %%%
%\addbibresource[label=other]{biblio/othercites.bib}
%\addbibresource[label=vak]{biblio/authorpapersVAK.bib}
%\addbibresource[label=papers]{biblio/authorpapers.bib}
%\addbibresource[label=conf]{biblio/authorconferences.bib}



%http://tex.stackexchange.com/a/141831/79756
%There is a way to automatically map the language field to the langid field. The following lines in the preamble should be enough to do that.
%This command will copy the language field into the  field and will then delete the contents of the language field. The language field will only be deleted if it was successfully copied into the langid field.
\DeclareSourcemap{ %модификация bib файла перед тем, как им займётся biblatex 
    \maps{
    	%% SPbPU
    	%% https://tex.stackexchange.com/a/229505/44348
    	\map{% delete month
    		\step[fieldset=month, null]
    		\step[fieldsource=date,
    		match=\regexp{([0-9]{4})-[0-9]{2}(-[0-9]{2})*},
    		replace=\regexp{$1}$5] % <<$5>> only for syntax highlihgting in IDE
    	}
%    	\map{% set current urldate
%    	\step[fieldset=urldate, null]	
%		\step[fieldset=urldate,fieldvalue={\the\year-\the\month-\the\day}]
%    	} 
%		\map{% не отображаем поле ``Глава''
%			\step[fieldset=chapter, null]
%			\step[fieldset=editor, null]
%		}
%    	} 
		\map{% перекидываем значения полей hyphenation в поля langid, которыми пользуется biblatex
			\step[fieldsource=hyphenation, fieldset=langid, origfieldval, final]
		}
        \map{% перекидываем значения полей language в поля langid, которыми пользуется biblatex
            \step[fieldsource=language, fieldset=langid, origfieldval, final]
            \step[fieldset=language, null]
        }
%        \map[overwrite, refsection=0]{% стираем значения всех полей addendum
%            \perdatasource{biblio/authorpapersVAK.bib}
%            \perdatasource{biblio/authorpapers.bib}
%            \perdatasource{biblio/authorconferences.bib}
%            \step[fieldsource=addendum, final]
%            \step[fieldset=addendum, null] %чтобы избавиться от информации об объёме авторских статей, в отличие от автореферата
%        }
        \map{% перекидываем значения полей numpages в поля pagetotal, которыми пользуется biblatex
            \step[fieldsource=numpages, fieldset=pagetotal, origfieldval, final]
            \step[fieldset=pagestotal, null]
        }
        \map{% если в поле medium написано "Электронный ресурс", то устанавливаем поле media, которым пользуется biblatex, в значение eresource.
            \step[fieldsource=medium,
            match=\regexp{Электронный\s+ресурс},
            final]
            \step[fieldset=media, fieldvalue=eresource]
        }
        \map[overwrite]{% стираем значения всех полей issn
            \step[fieldset=issn, null]
        }
        \map[overwrite]{% стираем значения всех полей abstract, поскольку ими не пользуемся, а там бывают "неприятные" латеху символы
            \step[fieldsource=abstract]
            \step[fieldset=abstract,null]
        }
        \map[overwrite]{ % переделка формата записи даты
            \step[fieldsource=urldate,
            match=\regexp{([0-9]{2})\.([0-9]{2})\.([0-9]{4})},
            replace={$3-$2-$1$4}, %, % $4 вставлен исключительно ради нормальной работы программ подсветки синтаксиса, которые некорректно обрабатывают $ в таких конструкциях
            final]
        }
%        \map[overwrite]{ % добавляем ключевые слова, чтобы различать источники
%            \perdatasource{biblio/othercites.bib}
%            \step[fieldset=keywords, fieldvalue={biblioother,bibliofull}]
%        }
%        \map[overwrite]{ % добавляем ключевые слова, чтобы различать источники
%            \perdatasource{biblio/authorpapersVAK.bib}
%            \step[fieldset=keywords, fieldvalue={biblioauthorvak,biblioauthor,bibliofull}]
%        }
%        \map[overwrite]{ % добавляем ключевые слова, чтобы различать источники
%            \perdatasource{biblio/authorpapers.bib}
%            \step[fieldset=keywords, fieldvalue={biblioauthornotvak,biblioauthor,bibliofull}]
%        }
%        \map[overwrite]{ % добавляем ключевые слова, чтобы различать источники
%            \perdatasource{biblio/authorconferences.bib}
%            \step[fieldset=keywords, fieldvalue={biblioauthorconf,biblioauthor,bibliofull}]
%        }
%        \map[overwrite]{% стираем значения всех полей series
%            \step[fieldset=series, null]
%        }
        \map[overwrite]{% перекидываем значения полей howpublished в поля organization для типа online
            \step[typesource=online, typetarget=online, final]
            \step[fieldsource=howpublished, fieldset=organization, origfieldval]
            \step[fieldset=howpublished, null]
        }
        % Так отключаем [Электронный ресурс]
%        \map[overwrite]{% стираем значения всех полей media=eresource
%            \step[fieldsource=media,
%            match={eresource},
%            final]
%            \step[fieldset=media, null]
%        }
		\map{
			\step[fieldsource=langid, match=russian, final]
			\step[fieldset=presort, fieldvalue={a}]
		}
		\map{
			\step[fieldsource=langid, notmatch=russian, final]
			\step[fieldset=presort, fieldvalue={z}]
		}%    
	}
}


%\DeclareSourcemap{
%	\maps[datatype=bibtex]{
%		\map{
%			\step[fieldsource=langid, match=russian, final]
%			\step[fieldset=presort, fieldvalue={a}]
%		}
%		\map{
%			\step[fieldsource=langid, notmatch=russian, final]
%			\step[fieldset=presort, fieldvalue={z}]
%		}
%	}
%}


%%% Убираем неразрывные пробелы перед двоеточием и точкой с запятой %%%
%\makeatletter
%\ifnumequal{\value{draft}}{0}{% Чистовик
%    \renewcommand*{\addcolondelim}{%
%      \begingroup%
%      \def\abx@colon{%
%        \ifdim\lastkern>\z@\unkern\fi%
%        \abx@puncthook{:}\space}%
%      \addcolon%
%      \endgroup}
%
%    \renewcommand*{\addsemicolondelim}{%
%      \begingroup%
%      \def\abx@semicolon{%
%        \ifdim\lastkern>\z@\unkern\fi%
%        \abx@puncthook{;}\space}%
%      \addsemicolon%
%      \endgroup}
%}{}
%\makeatother

%%% Правка записей типа thesis, чтобы дважды не писался автор
%\ifnumequal{\value{draft}}{0}{% Чистовик
%\DeclareBibliographyDriver{thesis}{%
%  \usebibmacro{bibindex}%
%  \usebibmacro{begentry}%
%  \usebibmacro{heading}%
%  \newunit
%  \usebibmacro{author}%
%  \setunit*{\labelnamepunct}%
%  \usebibmacro{thesistitle}%
%  \setunit{\respdelim}%
%  %\printnames[last-first:full]{author}%Вот эту строчку нужно убрать, чтобы автор диссертации не дублировался
%  \newunit\newblock
%  \printlist[semicolondelim]{specdata}%
%  \newunit
%  \usebibmacro{institution+location+date}%
%  \newunit\newblock
%  \usebibmacro{chapter+pages}%
%  \newunit
%  \printfield{pagetotal}%
%  \newunit\newblock
%  \usebibmacro{doi+eprint+url+note}%
%  \newunit\newblock
%  \usebibmacro{addendum+pubstate}%
%  \setunit{\bibpagerefpunct}\newblock
%  \usebibmacro{pageref}%
%  \newunit\newblock
%  \usebibmacro{related:init}%
%  \usebibmacro{related}%
%  \usebibmacro{finentry}}
%}{}

%\newbibmacro{string+doi}[1]{% новая макрокоманда на простановку ссылки на doi
%    \iffieldundef{doi}{#1}{\href{http://dx.doi.org/\thefield{doi}}{#1}}}

%\ifnumequal{\value{draft}}{0}{% Чистовик
%\renewcommand*{\mkgostheading}[1]{\usebibmacro{string+doi}{#1}} % ссылка на doi с авторов. стоящих впереди записи
%\renewcommand*{\mkgostheading}[1]{#1} % только лишь убираем курсив с авторов
%}{}
%\DeclareFieldFormat{title}{\usebibmacro{string+doi}{#1}} % ссылка на doi с названия работы
%\DeclareFieldFormat{journaltitle}{\usebibmacro{string+doi}{#1}} % ссылка на doi с названия журнала
%%% Убрать тире из разделителей элементов в библиографии:
%\renewcommand*{\newblockpunct}{%
%    \addperiod\space\bibsentence}%block punct.,\bibsentence is for vol,etc.

%%% Возвращаем запись «Режим доступа» %%%
%\DefineBibliographyStrings{english}{%
%    urlfrom = {Mode of access}
%}
%\DeclareFieldFormat{url}{\bibstring{urlfrom}\addcolon\space\url{#1}}

%%%% В списке литературы обозначение одной буквой диапазона страниц англоязычного источника %%%
\DefineBibliographyStrings{english}{%
    pages = {P\adddot} %заглавность буквы затем по месту определяется работой самого biblatex
}

%%% В ссылке на источник в основном тексте с указанием конкретной страницы обозначение одной большой буквой %%%
%\DefineBibliographyStrings{russian}{%
%    page = {C\adddot}
%}

%%% Исправление длины тире в диапазонах %%%
%\DefineBibliographyExtras{russian}{%
%  \protected\def\bibrangedash{%
%    \textendash\penalty\value{abbrvpenalty}}% almost unbreakable dash
%  \protected\def\bibdaterangesep{\bibrangedash}%тире для дат
%}

%Set higher penalty for breaking in number, dates and pages ranges
\setcounter{abbrvpenalty}{10000} % default is \hyphenpenalty which is 12

%Set higher penalty for breaking in names
\setcounter{highnamepenalty}{10000} % If you prefer the traditional BibTeX behavior (no linebreaks at highnamepenalty breakpoints), set it to ‘infinite’ (10 000 or higher).
\setcounter{lownamepenalty}{10000}

%%% Set low penalties for breaks at uppercase letters and lowercase letters
%\setcounter{biburllcpenalty}{500} %управляет разрывами ссылок после маленьких букв RTFM biburllcpenalty
%\setcounter{biburlucpenalty}{3000} %управляет разрывами ссылок после больших букв, RTFM biburlucpenalty

%% Список литературы с красной строки (без висячего отступа) %%%
%\printfield  ----  This command prints a hfieldi using the formatting directive hformati, as defined
%with \DeclareFieldFormat. 
%\printtext   ----  This command prints htexti, which may be printable text or arbitrary code generating
%printable text. It clears the punctuation buffer before inserting htexti and
%informs biblatex that printable text has been inserted.
%https://github.com/odomanov/biblatex-gost/blob/master/tex/latex/biblatex-gost/bbx/gost-numeric.bbx
\defbibenvironment{SSTfirst} % Примерно соответствует 1 варианту оформления списка использованных источников
  {\list
     {
    \printtext[labelnumberwidth]{%
	\printfield{labelprefix}%not numberprefix !
	\printfield{labelnumber}}
	}%
     {%
      \setlength{\labelwidth}{\labelnumberwidth}% This length register indicates the width of the widest labelnumber
      \addtolength{\labelwidth}{6pt}% \labelwidth Сдвигаем label, чтобы визуально сравнять с enumitem
      \setlength{\leftmargin}{\labelwidth}% default is \labelwidth используют также \parindent в enumerations
      \setlength{\labelsep}{\widthof{\  }}% Управляет длиной отступа после точки % default is \biblabelsep
      \setlength{\leftskip}{-2.2em}
      \addtolength{\leftmargin}{\leftskip}%<----- here
      \setlength{\itemsep}{0pt}% Управление дополнительным вертикальным разрывом между записями. \bibitemsep по умолчанию соответствует \itemsep списков в документе.
      \setlength{\itemindent}{\bibhang}% Пользуемся тем, что \bibhang по умолчанию принимает значение \parindent (абзацного отступа), который переназначен в styles.tex
      \addtolength{\itemindent}{\labelwidth}% \labelwidth Сдвигаем правее на величину номера с точкой
      \addtolength{\itemindent}{\labelsep}% \labelsep Сдвигаем ещё правее на отступ после точки
      \setlength{\parsep}{\bibparsep}% расстояние между параграфами 
     }%
      \renewcommand*{\makelabel}[1]{\hss##1} % выравнивание по labelnumberwidth >= 2 строки item
  }
  {\endlist}
  {\item}
% % % %
\defbibenvironment{tsk} % переопределяем окружение библиографии из gost-numeric.bbx пакета biblatex-gost
  {\list
	{
		\printtext[labelnumberwidth]{%
			\printfield{labelprefix}%not numberprefix !
			\printfield{labelnumber}}
	}%
	{%
		\setlength{\labelwidth}{\labelnumberwidth}% This length register indicates the width of the widest labelnumber
		%      \addtolength{\labelwidth}{-3pt}% \labelwidth Сдвигаем label, чтобы визуально сравнять с enumitem
		\setlength{\leftmargin}{\labelwidth}% default is \labelwidth используют также \parindent в enumerations
		\setlength{\labelsep}{\widthof{\  }}% Управляет длиной отступа после точки % default is \biblabelsep
		\setlength{\itemsep}{0pt}% Управление дополнительным вертикальным разрывом между записями. \bibitemsep по умолчанию соответствует \itemsep списков в документе.
		\setlength{\itemindent}{0pt}% Пользуемся тем, что \bibhang по умолчанию принимает значение \parindent (абзацного отступа), который переназначен в styles.tex
		\addtolength{\itemindent}{0pt}% \labelwidth Сдвигаем правее на величину номера с точкой
		\addtolength{\itemindent}{0pt}% \labelsep Сдвигаем ещё правее на отступ после точки
		\setlength{\parsep}{\bibparsep}% расстояние между параграфами 
	}%
	\renewcommand*{\makelabel}[1]{\hss##1} % выравнивание по labelnumberwidth >= 2 строки item
}
{\endlist}
{\item}

%%% Счётчик использованных ссылок на литературу, обрабатывающий с учётом неоднократных ссылок
%%http://tex.stackexchange.com/a/66851/79756
%\newcounter{citenum}
%\newtotcounter{citenum}
%\makeatletter
%\defbibenvironment{counter} %Env of bibliography
%  {\setcounter{citenum}{0}%
%  \renewcommand{\blx@driver}[1]{}%
%  } %what is doing at the beginining of bibliography. In your case it's : a. Reset counter b. Say to print nothing when a entry is tested.
%  {} %Здесь то, что будет выводиться командой \printbibliography. \thecitenum сюда писать не надо
%  {\stepcounter{citenum}} %What is printing / executed at each entry.
%\makeatother
%\defbibheading{counter}{}

%
%
%\newtotcounter{citeauthorvak}
%\makeatletter
%\defbibenvironment{countauthorvak} %Env of bibliography
%{\setcounter{citeauthorvak}{0}%
%    \renewcommand{\blx@driver}[1]{}%
%} %what is doing at the beginining of bibliography. In your case it's : a. Reset counter b. Say to print nothing when a entry is tested.
%{} %Здесь то, что будет выводиться командой \printbibliography. Обойдёмся без \theciteauthorvak в нашей реализации
%{\stepcounter{citeauthorvak}} %What is printing / executed at each entry.
%\makeatother
%\defbibheading{countauthorvak}{}
%
%\newtotcounter{citeauthornotvak}
%\makeatletter
%\defbibenvironment{countauthornotvak} %Env of bibliography
%{\setcounter{citeauthornotvak}{0}%
%    \renewcommand{\blx@driver}[1]{}%
%} %what is doing at the beginining of bibliography. In your case it's : a. Reset counter b. Say to print nothing when a entry is tested.
%{} %Здесь то, что будет выводиться командой \printbibliography. Обойдёмся без \theciteauthornotvak в нашей реализации
%{\stepcounter{citeauthornotvak}} %What is printing / executed at each entry.
%\makeatother
%\defbibheading{countauthornotvak}{}
%
%\newtotcounter{citeauthorconf}
%\makeatletter
%\defbibenvironment{countauthorconf} %Env of bibliography
%{\setcounter{citeauthorconf}{0}%
%    \renewcommand{\blx@driver}[1]{}%
%} %what is doing at the beginining of bibliography. In your case it's : a. Reset counter b. Say to print nothing when a entry is tested.
%{} %Здесь то, что будет выводиться командой \printbibliography. Обойдёмся без \theciteauthorconf в нашей реализации
%{\stepcounter{citeauthorconf}} %What is printing / executed at each entry.
%\makeatother
%\defbibheading{countauthorconf}{}
%
%\newtotcounter{citeauthor}
%\makeatletter
%\defbibenvironment{countauthor} %Env of bibliography
%{\setcounter{citeauthor}{0}%
%    \renewcommand{\blx@driver}[1]{}%
%} %what is doing at the beginining of bibliography. In your case it's : a. Reset counter b. Say to print nothing when a entry is tested.
%{} %Здесь то, что будет выводиться командой \printbibliography. Обойдёмся без \theciteauthor в нашей реализации
%{\stepcounter{citeauthor}} %What is printing / executed at each entry.
%\makeatother
%\defbibheading{countauthor}{}
%
%\defbibheading{authorpublications}[\authorbibtitle]{\section*{#1}}
%\defbibheading{pubsubgroup}{\noindent\textbf{#1}}
%\defbibheading{otherpublications}{\section*{#1}}


%%%% Создание команд для вывода списка литературы %%%
%\newcommand*{\insertbibliofull}{
%\printbibliography[keyword=bibliofull,section=0]
%\printbibliography[heading=counter,env=counter,keyword=bibliofull,section=0]
%}
%
%\newcommand*{\insertbiblioauthor}{
%\printbibliography[heading=authorpublications,keyword=biblioauthor,section=1,title=\authorbibtitle]
%}
%\newcommand*{\insertbiblioauthorimportant}{
%\printbibliography[heading=authorpublications,keyword=biblioauthor,section=2,title={Наиболее значимые \MakeLowercase{\protect\authorbibtitle{}}}]
%}
%\newcommand*{\insertbiblioauthorgrouped}{% Заготовка для вывода сгруппированных печатных работ автора. Порядок нумерации определяется в соответствующих счетчиках внутри окружения refsection в файле common/characteristic.tex
%\section*{\authorbibtitle}
%\printbibliography[heading=pubsubgroup, keyword=biblioauthorvak, section=1,title=\vakbibtitle]%
%\printbibliography[heading=pubsubgroup, keyword=biblioauthorconf, section=1,title=\confbibtitle]%
%\printbibliography[heading=pubsubgroup, keyword=biblioauthornotvak, section=1,title=\notvakbibtitle]%
%}
%
%\newcommand*{\insertbiblioother}{
%\printbibliography[heading=otherpublications,keyword=biblioother]
%}




%Bibliography update
%TO delete predefined description of References
\defbibheading{bibliography}[\bibname]{%
	\markboth{#1}{#1}}
%https://tex.stackexchange.com/a/307757/44348
\newcommand{\fullbibtitle}{Библиографический список} % (ГОСТ Р 7.0.11-2011, 4)


%%%delete / in urldate (works together with \map)
\NewBibliographyString{urldateprefix}
%
\DefineBibliographyStrings{russian}{%
	urldateprefix = {дата обращения\addcolon}}
\DefineBibliographyStrings{english}{%
	urldateprefix = {visited on}}
\DeclareFieldFormat{urldate}{(\bibstring{urldateprefix}\addspace\mkdayzeros{\thefield{urlday}}\adddot\mkmonthzeros{\thefield{urlmonth}}\adddot\mkyearzeros{\thefield{urlyear}})}




%% small ``p'' before total page number of books can be made automatically only by 
%%


%renew (chapter+pages) to series format
%\newbibmacro*{chapter+pages}{%
%	\iffieldundef{chapter}
%	{}%
%	{\printfield{chapter}\space---\space}% 
%%	\setunit*{\bibpagespunct}%
%	\printfield{pages}%
%	\newunit}
\newbibmacro*{chapter+pages}{%
	\iffieldundef{chapter}
	{}%
	{\printfield{chapter}%
		\setunit*{\addspace---\space}}%
	\printfield{pages}%
	\newunit}



%delete / from date format - REPLACED BY \map to delete totally
%\DeclareFieldFormat{date}{\thefield{year}}
%TO-DO: add month with condition if it does not exist than do not write the dot
%\thefield{month}\nobreakdash\adddot

%delete : from DOI format 
\DeclareFieldFormat{doi}{%
	\mkbibacro{DOI}\space
	\ifhyperref
	{\href{https://doi.org/#1}{\nolinkurl{#1}}}
	{\nolinkurl{#1}}\isdot} 


%%%add Ser.: to series format
%% First approach
\NewBibliographyString{seriesprefix}
%
\DefineBibliographyStrings{english}{%
	seriesprefix = {Ser}}
\DefineBibliographyStrings{russian}{%
	seriesprefix = {Сер}}
\DeclareFieldFormat{series}{\bibstring{seriesprefix}\adddot\addcolon\space{#1}\isdot}

%% Second approach % nested conditions
%\DeclareFieldFormat{series}{\iffieldequalstr{langid}{russian}{Сер}{\iffieldequalstr{langid}{english}{Bingo}{NotSpecified}}\adddot\addcolon\space{#1}\isdot} %


%add brackets to note format
\DeclareFieldFormat{note}{\mkbibparens{{#1}}\isdot} 

%delete spaces before ; and :
\renewcommand*{\addcolondelim}{\addcolon\space}
\renewcommand*{\addsemicolondelim}{\addsemicolon\space}

%modify PhD theis, for candidate one can use words CANDthesis
\DefineBibliographyStrings{english}{phdthesis = {diss\adddot\space\ldots}}
\renewcommand*{\gostmedialanguage}{}